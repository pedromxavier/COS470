\documentclass[brazil]{homework}
\title{Sistemas Distribuídos}
\subtitle{COS470 - Lista 1}
\author{Pedro Maciel Xavier}
\register{116023847}
\date{22 de julho de 2021}

\begin{document}
\maketitle*

\section*{Sistemas de Computação}

\textbf{Descrição}

Os Sistemas Distribuídos de Computação tem como finalidade desempenhar tarefas computacionais intensivas como, por exemplo, processar grandes quantidades de dados, operar simulações ou buscar pelas soluções ótimas de um problema. A distribuição das tarefas proporciona escalabilidade e performance, unindo diversos computadores nas tarefas que se deseja realizar. Estes sistemas são majoritariamente utilizados em cenários de execução paralela, onde é possível dividir a carga de trabalho de maneira concorrente entre as máquinas envolvidas.

\textbf{Particularidades}

Podemos classificar os Sistemas de Computação em duas categorias segundo sua arquitetura. Sistemas onde os computadores participantes possuem configurações semelhantes e estão conectados por redes locais de alta velocidade são conhecidos como \textit{Clusters}. Estes sistemas, muitas vezes chamados de supercomputadores, se encontram em organizações onde existe uma demanda recorrente por desempenho no processamento intensivo.

Outro paradigma é encontrado nas arquiteturas em \textit{Grid}. Nesse caso, temos um sistema heterogêneo, isto é, que agrega uma miríade de computadores operando sob configurações diversas, conectados pela internet, e que realizam, cada um, uma pequena parte da tarefa proposta. Um exemplo de implementação feita na década passada permitia que pessoas ao redor do mundo dedicassem parte da capacidade de processamento de seus computadores pessoais para um sistema em \textit{Grid} cujo objetivo era realizar simulações de interações farmacológicas, elencando substâncias candidatas ao tratamento do câncer.

\section*{Sistemas de Informação}

\textbf{Descrição}

Já os Sistemas Distribuídos de Informação, por sua vez, são uma classe ainda mais abrangente de sistemas. São caracterizados desta forma sistemas que manipulam, armazenam e utilizam dados, providenciando serviços que operam sobre estas informações. São sistemas distribuídos de informação alguns dos recursos computacionais mais importantes da atualidade, como o sistema bancário e de varejo digital, por exemplo.

\textbf{Particularidades}

A distribuição das tarefas entre diversos computadores hospedados em diferentes locais permite atender um número muito maior de pedidos dos usuários simultâneamente e ainda assim garantir uma maior disponibilidade do serviço, ou seja, uma maior tolerância a falhas. Além do comércio e dos bancos, notáveis sistemas de informação estão presentes no campo do entretenimento e da comunicação. Serviços de \textit{streaming} e de chamdas em vídeo são capazes de transmitir filmes para milhões de usuários ao mesmo tempo e conectar também um número gigantesco de pessoas ao redor do mundo de maneira eficiente. Tornar isso possível envolve um esforço considerável no balanceamento de carga da rede e na garantia da disponibilidade de determinados recursos em tempo real.

As principais características destes sistemas residem, portanto, na capacidade de prover serviços para muitos usuários simultâneos de maneira transparente e atômica. Quem utiliza o sistema não tem conhecimento de que entidade será responsável por processar o seu pedido e, da mesma forma, pedidos conflitantes emitidos concomitantemente devem ser isolados, a fim de preservar a integridade das informações relacionadas.

\subsection*{Arquitetura em Nuvem}

Tanto os Sistemas de Computação quanto os de Informação se encontram, atualmente, muito presentes em ambientes de arquitetura em Nuvem, que tem se tornado uma opção muito popular por providenciar os recursos computacionais necessários para a operação dos sistemas como fossem serviços. 

\section*{Sistemas Pervasivos}

\textbf{Descrição}

Os Sistemas Pervasivos são compostos, em sua maioria, por uma enorme quantidade de dispositivos simples e de baixo custo, como microcontroladores e sensores. Dispositivos móveis e \textit{wearables} também fazem parte deste ambiente. De maneira oposta ao prisma dos sistemas enunciados anteriormente, estes estão presentes em ambientes que compartilhamos em nosso cotidiano.

\textbf{Particularidades}

Este que é, dentre os três tipos, o mais diverso considerando a fauna de dispositivos utilizados, o mais heterogêneo e geograficamente disperso agregado de sistemas, tem por característica estar presente em todos os ambientes possíveis, desde os principais palcos da vida urbana até mesmo nas atividades rotineiras de cultivo em áreas rurais. Termos como \textit{ubiquidade} e \textit{onipresença} são recorrentemente utilizados para tratar desta realidade.

Fora do contexto da internet, onde participam diversos sistemas pervasivos, encontramos de maneira independente, mas internamente conectada, as redes de sensores. São sistemas numerosos que contam com grandes malhas de sensores integrados e autônomos, capazes de tomar decisões em conjunto. Aparecem dispersos em lavouras aferindo indicadores ambientais e ordenando a rega, assim como em estruturas prediais modernas sendo capazes de identificar e alertar sobre o início de um incêndio.

\end{document}