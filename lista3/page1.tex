\quest*{} %1
\begin{answer}
    A arquitetura cliente-servidor pressupõe a existência de papéis distintos no sistema distribuído. Em geral, o cliente é um ator que demanda serviços enquanto o servidor se encarrega de prover tais serviços.
\end{answer}

\quest*{} %2
\begin{answer}
    
\end{answer}

\quest*{} %3
\begin{answer}
    
\end{answer}

\quest*{} %4
\begin{answer}
    
\end{answer}

\quest*{} %5
\begin{answer}
    
\end{answer}

\quest*{} %6
\begin{answer}
    
\end{answer}

\quest*{} %7
\begin{answer}
    
\end{answer}

\quest*{} %8
\begin{answer}
    
\end{answer}

\quest*{} %9
\begin{answer}
    
\end{answer}

\quest*{} %10
\begin{answer}
    
\end{answer}

\quest*{} %11
\begin{answer}
    
\end{answer}

\quest*{} %12
\begin{answer}
    
\end{answer}